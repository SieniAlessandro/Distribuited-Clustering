\chapter{REST Server}
Per integrare gli algoritmi di clustering realizzati in python con il core del progetto realizzato invece in Java è stato di scelto di far comunicare i due linguaggi
mediante l'utilizzo di un serve REST in grado di fornire le funzioni realizzate in python tramite messaggi REST appositamente realizzati.
\section{Implementazione}
Dal punto di vista implementativo è stato scelto di realizzare il server REST mediante l'utilizzo della libreria Flask, in quanto offre un servizio completamente funzionante e
modificabile seguendo le preferenze del programmatore.
\subsection{Gestione delle richieste}
Per poter chiamare i metodi messi a disposizione dal server REST vengono effettuate delle richieste REST nella quale si specifica, mediante un messaggio json, il metodo da richiamare ed gli argomenti necessari, rimanendo
che il risultato venga processato e che un codice corrispondente allo stato d'esecuzione del metodo venga rispedito al mittente
\lstinputlisting[language=Python]{../../src/Python/RestServer.py}
